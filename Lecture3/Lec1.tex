\documentclass[]{article}

\usepackage{amssymb,latexsym,amsmath,amsthm}     % Standard packages


%%%%%%%%%%%
% Margins %
%%%%%%%%%%%
\addtolength{\textwidth}{1.0in}
\addtolength{\textheight}{1.00in}
\addtolength{\evensidemargin}{-0.75in}
\addtolength{\oddsidemargin}{-0.75in}
\addtolength{\topmargin}{-.50in}


%%%%%%%%%%%%%%%%%%%%%%%%%%%%%%%%
% Theorem/Proof Environments %
%%%%%%%%%%%%%%%%%%%%%%%%%%%%%%
\newtheorem*{myax}{Axiom}
\newtheorem*{mynot}{Notation}
\newtheorem{theorem}{Theorem}[section]
\newtheorem{myrem}{Remark}[section]
\newtheorem{mydef}{Definition}[section]


% Document %
\begin{document}

\begin{flushright}
	\textit{InfoViz, Michael Hirsch, Feb $9^{th}$ 2015}
\end{flushright}

\section{Graphs Networks and Trees}

Mostly from: Visual Analysis of Large Graphs

...recall Schneiderman Data Types.

A \textbf{graph} is a set of nodes with a set of edges (relations) between them. We have directed graphs and undirected graphs. Nodes and edges can also be characterized differently using edge/node types. Colors, shapes, etc. The nodes and edges can also have different weights, size is a good indicator of this. A \textbf{tree} is a special case of a graph in which any two of the nodes are connected by exactly one path. No cycles also. 

In graph theory, a network is a directed graph with weighted edge. In info viz, a network is any graph with attributes associated to nodes and edges. A minimum spanning tree is a subset of a graph connecting all vertices of a graph. \textbf{compound graphs} are a way of grouping nodes together to create a tree structure.

\textbf{Dynamic Graphs}. These can evolve over time, with changes occuring in atttributes of nodes, edges, structure, or combinations of these. The \textit{data} is changing. 

\textbf{Topological properties:} graph size - the number of nodes. Graph density - number of edges. Sparse graphs have few edges, complete graphs have everything connected, cliques are a subset of a graph which is fully connected.

So what is a large graph? In algorithmic analysis, this means long computation times or a large memory footprint. In info viz, this means anything leading to cluttered displays. A question we ask is \textit{how can we reduce the size of our graph?}.

We look at the degree of the notes. Graph filtering: stochastic sampling - randomly remove nodes or edges from graph. Geodesic clustering: looks at structure of graph and tries to keep connection information and throwing away duplicate information. Structure based filtering: remove any edges so long as elements remain connected.	

\textit{visual representation of a graph}. where do we put the nodes, how do we draw the nodes, and how to we display the edges?

Is our graph planar? That is, can we draw the graphs in such a way to ensure that no edges overlap. There are algorithms for finding these, and then algos for fixing. We have aesthetic rules: Reduce visual clutter, reduce spatial aliases, spatial matching of multiple representatives, maximize compactness. For dynamic data: preserving the mental map, reducing the cognitive load, minimizing temporal aliases. Scalability, in number of vertices, edges, and graphs.

Predictability. Two different runs of the same algorithm on the same graph should yield identical results. 

Node-link diagrams: for geo data, each node has a position which is determined by the data. For other data, we are free to place the points in any position. Optimal positions are determined by the various aesthetic criteria. 
\end{document}